\documentclass[compress]{beamer}
\usepackage[utf8]{inputenc}
\usepackage{default}
\usepackage[T1]{fontenc}
\usepackage[spanish]{babel}
\usepackage{graphicx}
\usepackage{subfigure}
\usepackage{wrapfig} %Figuras al lado de texto
\usepackage[rflt]{floatflt} %Figuras flotantes entre el texto
\usepackage{eso-pic}
\usepackage[absolute,overlay]{textpos}
\usepackage{calc}
\usepackage{tikz}
\usepackage{longtable,multirow,booktabs,titling}
\usepackage{tikz}
\usepackage{verbatim}
\usepackage{multimedia}
%\usepackage{movie15}
%\usepackage{media9}
%\usepackage{hyperref}
%\usepackage[active,tightpage]{}


%\usepackage{xetex}
%% del paquete raro
%\usepackage{fontspec}
%\usepackage{xunicode}
%\usepackage{xltxtra}
%\setmainfont{Gill Sans}
%\setmonofont[Scale=0.86]{Andale Mono}
\uselanguage{spanish}
\languagepath{spanish}
\deftranslation[to=spanish]{WHO?}{¿Quién?}
\deftranslation[to=spanish]{WHEN?}{¿Cuando?}
\definecolor{beamer@solarized@green}{HTML}{859900}
%\usefonttheme{professionalfonts}
\usefonttheme{serif}
\setbeamercovered{transparent}
%\usetheme{CambridgeUS}
%\usetheme{Bergen} %me gusta
\usetheme{JuanLesPins}
%\usetheme{Berkeley} %me gusta
%\usetheme{Goettingen}
%\usetheme{Hannover} % me gusta
%\usetheme{Szeged}
%\usetheme{Singapore}
%\usetheme{AnnArbor}
\setbeamercolor{titlelike}{parent=structure} %%% Para quitarle el color de fondo
%\setbeamercolor{block title}{fg=black,bg=gray!20}
%\usetheme[width=1cm]{Bergen} %para modificar tamaño de la barra lateral
%\def\insertimage{\includegraphics[width= 1cm]{imagenes/escudob.png}}
%\def\insertauthorindicator{¿Quién?}% Default is "Who?"
%\def\insertinstituteindicator{¿Donde?}% Default is "From?"
%\def\insertdateindicator{¿Cuando?}% Default is "When?"
\usecolortheme[RGB={46,139,87}]{structure}
%\useoutertheme{shadow}
\useoutertheme[]{miniframes}
\useinnertheme{circles}
\usefonttheme[onlylarge]{structuresmallcapsserif}
\setbeamerfont{title}{shape=\itshape,family=\rmfamily}
%\addtobeamertemplate{sidebar right}{}{
 % \includegraphics[width=1.5cm,height=1.5cm]{imagenes/escudobt1.png}\\[0.5cm]
  %\includegraphics[width=2cm,height=1cm]{imagenes/escudobit.png}
%}
\usetikzlibrary{trees}
\usetikzlibrary{positioning}

%\setbeamersize{sidebar width left=12pt}

%\makeatletter
%\addtobeamertemplate{sidebar left}{}{%
  %\includegraphics[width=\beamer@sidebarwidth]{imagenes/escudob.png}%
%}
%\makeatother
%\logo{r}{\includegraphics[width=1.2cm,keepaspectratio]{imagenes/escudob.png}}
%con esto anulo la barra lateral
%\makeatletter
%\setbeamertemplate{sidebar canvas left}{}
%\setbeamertemplate{sidebar left}{%
 % \vspace*{\fill}
  %\includegraphics[width=1.5]{imagenes/escudob.png}
  %\vspace*{\fill}
%}
%\makeatother
%\setbeamercolor{structure}{fg=green!80!black,bg=green!20!white}

%\vspace{-2cm}
\hspace{-5.5cm}
\title{\Large{Importancia de las áreas verdes \\ 
de la ciudad de Puebla para la avifauna}}\\[1cm]
\author{\small{Fernando Dorantes Nieto}\\[1cm]
Director de tesis\\  \scriptsize{Dr Juan Héctor García Chávez}}\\[-2cm]
\date{}
\usebackgroundtemplate { %
\includegraphics[ width=\paperwidth,height=\paperheight] {imagenes/dibujpres12.png}}
%\includegraphics[ width=\paperwidth,height=\paperheight] {imagenes/pichochos.png}}

%\titlegraphic{\includegraphics[width=\textwidth,height=.5\textheight]{imagenes/escudob.png}}
%\logo{\includegraphics[width=1.5cm]{imagenes/escudobi.png}}
%\logo[]{%
%\centering  
%\makebox[0.95\paperwidth]{%
    %\includegraphics[width=1.5cm,keepaspectratio]{imagenes/escudob.png}%
    %\hfill%
    %\includegraphics[width=1.5cm,keepaspectratio]{imagenes/escudobi.png}%
  %}
%}
\section{Tesis}
\begin{document}
\begin{frame}
\begin{wrapfigure}{c}{2cm}
%\includegraphics[width= 1cm]{imagenes/escudob.png}
\end{wrapfigure} 
\begin{wrapfigure}{r}{0.5cm}
\includegraphics[width= 1.5cm]{imagenes/escudos.png}
\end{wrapfigure}
\titlepage
\vspace{-1cm}
%\begin{enumerate}
 %\item \tiny{Tesis que para presentar título de biólogo}
% \item \tiny{Desarrollado con \LaTeX}
%\end{enumerate}
\end{frame}
\fontsize{10}{2}
\setbeamertemplate{logo}{}
\section{Introducción}
{
\usebackgroundtemplate{\includegraphics[width=\paperwidth,height=\paperheight]{imagenes/dibujpres11.png}}
\begin{frame}
 \frametitle{Introducción}
\vspace{-0.5cm}
\begin{columns}[c]
\column{.6\textwidth}
\visible<2,3>{\begin{center}
\includegraphics[scale=0.25]{imagenes/barrau.png}
\end{center}}
\column{.6\textwidth}<3>
\visible<2,3>{\begin{center}
\includegraphics[scale=0.25]{imagenes/barrar.png}
\end{center}}
\end{columns}

\begin{center}
\hspace*{-0.5cm}\includegraphics[scale=0.31]{imagenes/barrasd.png}
\end{center}
\end{frame}
}

%{
%\usebackgroundtemplate{\includegraphics[width=\paperwidth,height=\paperheight]{imagenes/dibujpres11.png}}

%\begin{frame}
 %\frametitle{Introducción}
 %\begin{itemize}
%\tiny{  \item  El crecimiento de las ciudades representa un gran problema ambiental.\\
 % Teniendo como consecuencia otros problemas como la disminución de la biodiversidad y el cambio de uso de suelo.
 %\pause
  %\tiny{\item El gran crecimiento de las ciudades se puede ver en todos lados.\\
  %De acuerdo con la ONU (United Nations, 2014), el 54\% de la población humana vive en ciudades.\\}}
  %\end{itemize}
%\begin{center}
%\includegraphics[scale=0.08]{imagenes/siluetashumanas.png}

%\end{center}

%\end{frame}
%}
{
\usebackgroundtemplate{\includegraphics[width=\paperwidth,height=\paperheight]{imagenes/dibujpres11.png}}

\begin{frame}
\frametitle{Grandes regiones asfaltadas}
%\begin{itemize}
%\tiny{ \item Se conocen en el país 377  centros urbanos. La región central del país es la más densamente poblada.\\
 %Dicha región está representada por las ciudades de: México, Toluca, Puebla, Cuernavaca, Querétaro y Pachuca.
%\pause
%\tiny{\item Para el caso de la ciudad de Puebla, la situación no es muy diferente, encontrando un alto grado de crecimiento}}\\[-1cm]

%\end{itemize}\\[-1cm]
\vspace{-1cm}
\begin{center}
\hspace*{-0.5cm}\includegraphics[scale=0.3]{imagenes/Internet/mapa1.jpg}

\end{center}

\end{frame}
}
{
\usebackgroundtemplate{\includegraphics[width=\paperwidth,height=\paperheight]{imagenes/dibujpres11.png}}
\begin{frame}
\frametitle{Sitios hostiles}
\vspace{-1cm}

\begin{center}
\hspace*{-1cm}\includegraphics[scale=0.33]{imagenes/barrasd2.png}\\

\end{center}
\vspace{-0.5cm}
\begin{columns}
\column{0.5\textwidth}
\visible<2,3>{\begin{center}
\hspace*{-0.8cm}\includegraphics[scale=0.2]{imagenes/lineas1.png}\\

\end{center}}
\column{0.5\textwidth}
\visible<2,3>{\begin{center}
\hspace*{-0.5cm}\includegraphics[scale=0.2]{imagenes/lineas.png}\\

\end{center}}

\end{columns}
\end{frame}
}

%{
%\usebackgroundtemplate{\includegraphics[width=\paperwidth,height=\paperheight]{imagenes/dibujpres11.png}}
%\begin{frame}
%\frametitle{No todo es hostilidad}
%\vspace{-1.5cm}
%\begin{center}
%\hspace*{-1cm}\includegraphics[scale=0.27]{imagenes/barrasd3.png}\\

%\end{center}

%\end{frame}
%}


{
\usebackgroundtemplate{\includegraphics[width=\paperwidth,height=\paperheight]{imagenes/dibujpres11.png}}
\begin{frame}
\frametitle{No todo es hostilidad}
\vspace{-1.5cm}

\begin{center}
\hspace*{-1cm}\includegraphics[width=12.8cm,height=5cm]{imagenes/barrasd4.png}\\

\end{center}

\end{frame}
}

%{
%\usebackgroundtemplate{\includegraphics[width=\paperwidth,height=\paperheight]{imagenes/dibujpres11.png}}
%\begin{frame}
%\frametitle{}
%\begin{itemize}
%\small{\item No todo es hostilidad}
%\pause
%\tiny{ \item Sin embargo las ciudades tienen todos los atributos propios de un ecosistema.}
%\pause
%\tiny{\item Comprenden una matriz de concreto y asfalto alternada por cambios abruptos que pueden ser las áreas verdes, camellones, terrenos baldíos etc.}
%\pause
%\tiny{\item Estudios han demostrado que dichas áreas verdes son sitios atractivos para diversos organismos}

%\end{itemize}\\[-1cm]

%\end{frame}
%}

%{
%\usebackgroundtemplate{\includegraphics[width=\paperwidth,height=\paperheight]{imagenes/dibujpres11.png}}
%\begin{frame}
%\frametitle{Ecología urbana}
%\vspace{-0.5cm}

%\begin{center}
%\hspace*{-1cm}\includegraphics[scale=0.25]{imagenes/mapa1pres.png}\\

%\end{center}

%\end{frame}
%}


{
\usebackgroundtemplate{\includegraphics[width=\paperwidth,height=\paperheight]{imagenes/dibujpres11.png}}
\begin{frame}
\frametitle{¿Qué falta por hacer?}
\vspace{-0.5cm}

\begin{center}
\hspace*{-1cm}\includegraphics[scale=0.29]{imagenes/barrasd5.png}\\

\end{center}

\end{frame}
}

{
\usebackgroundtemplate{\includegraphics[width=\paperwidth,height=\paperheight]{imagenes/dibujpres11.png}}
\begin{frame}
\frametitle{¿Qué falta por hacer?}
\vspace{-0.5cm}

\begin{center}
\hspace*{-1cm}\includegraphics[scale=0.29]{imagenes/barrasd6.png}\\

\end{center}

\end{frame}
}

{
\usebackgroundtemplate{\includegraphics[width=\paperwidth,height=\paperheight]{imagenes/dibujpres11.png}}
\begin{frame}
\frametitle{¿Qué falta por hacer?}
\vspace{-0.5cm}

\begin{center}
\hspace*{-1cm}\includegraphics[scale=0.29]{imagenes/barrasdn.png}\\

\end{center}

\end{frame}
}


%{
%\usebackgroundtemplate{\includegraphics[width=\paperwidth,height=\paperheight]{imagenes/dibujpres11.png}}
%\begin{frame}
%\frametitle{¿Qué falta por hacer?}
%\vspace{-0.5cm}

%\begin{center}
%\hspace*{-1cm}\includegraphics[scale=0.25]{imagenes/barrasd7.png}\\

%\end{center}

%\end{frame}
%}

%{
%\usebackgroundtemplate{\includegraphics[width=\paperwidth,height=\paperheight]{imagenes/dibujpres11.png}}
%\begin{frame}
%\frametitle{}
%\begin{itemize}
%\small{\item Ecología Urbana}
%\pause
%\tiny{ \item Las áreas verdes son los hábitats más parecidos que encontramos en zonas rurales  en cuanto a la estructura de la vegetación.}
%\pause
%\tiny{\item Se han realizado diversos estudios en varios estados del país.}
%\pause
%\tiny{\item Sin embargo la gran mayoría de los estudios solo toma en cuenta las áreas verdes ignorando el resto de las ciudades.}

%\end{itemize}\\[-1cm]

%\end{frame}
%}

{
\usebackgroundtemplate{\includegraphics[width=\paperwidth,height=\paperheight]{imagenes/dibujpres11.png}}
\begin{frame}
\frametitle{Objetivos}
\vspace{-1.5cm}
\begin{enumerate}
\normalsize{ \item Describir la comunidad de aves de la ciudad de Puebla.}\\[1cm]
\pause
\normalsize{\item Comparar la comunidad de aves de las áreas verdes con las áreas grises adyacentes a dichas áreas verdes, por medio de índices de diversidad $\alpha$ y $\beta$.}\\[1cm]
\pause
\normalsize{\item  Determinar la importancia relativa de las áreas verdes como sitios  refugio para las aves.}\\[0.5cm]
\pause
\normalsize{ \item Comparar la estructura  y riqueza de los gremios de las áreas verdes y grises de la ciudad de Puebla.}\\[1cm]

\end{enumerate}\\[-1cm]

\end{frame}
}



\section{Material y método}
{
\usebackgroundtemplate{\includegraphics[width=\paperwidth,height=\paperheight]{imagenes/dibujpres11.png}}
\begin{frame}
\frametitle{Material y método}
\begin{columns}[c]
%\column{.7\textwidth}
\column{.5\textwidth}

\vspace{-1cm}
\begin{center}
%\hspace*{-1cm}\includegraphics[scale=0.36]{imagenes/mapapueblagen.png}\\
\hspace*{-0.5cm}\includegraphics[scale=0.4]{imagenes/mapapuez1.png}\\

\end{center}
\pause
\column{.4\textwidth}
\vspace{-2.3cm}
\begin{itemize}
\small{  \item   Altitud: 2160 m\\
Temperatura: 17$^{\circ}$C\\}\\[1cm]
\pause
\small{ \item Clima templado subhúmedo\\
Precipitación: 900mm}\\[1cm]
\pause
\small{ \item Área: 230 Km$^{2}$.}\\[1cm]
\pause
\small{ \item  1,539,839 habitantes.}
\end{itemize}
\end{columns}
\end{frame}
}
%{
%\usebackgroundtemplate{\includegraphics[width=\paperwidth,height=\paperheight]{imagenes/dibujpres11.png}}
%\begin{frame}
%\frametitle{}
%\vspace{-0.5cm}

%\begin{center}
%\includegraphics[scale=0.27]{imagenes/Internet/strongs.jpg}\\

%\end{center}

%\end{frame}
%}

{
\usebackgroundtemplate{\includegraphics[width=\paperwidth,height=\paperheight]{imagenes/dibujpres11.png}}

\begin{frame}
\frametitle{Áreas verdes}
\begin{center}
\includegraphics[scale=0.32]{imagenes/mapaareaspres.png}\\
\end{center}
%{\footnotesize
%\begin{longtable}[c] {lc} 
%\caption{} \\ \toprule
%Área verde & Área (hectáreas)  \\ \midrule
%Parque Laguna de Chapulco  & 18   \\
%Ciudad universitaria BUAP & 106 \\
%Los Fuertes & 104  \\
%Panteón Municipal  & 19  \\
%Parque del Arte  & 8  \\
%Parque del rio Atoyac & 55  \\
%Parque Ecológico  & 65  \\
%Parque Juárez  & 6  \\
%Parque Paseo Bravo & 5  \\ \bottomrule
%\end{longtable}
%}
\end{frame}
}

{
\usebackgroundtemplate{\includegraphics[width=\paperwidth,height=\paperheight]{imagenes/dibujpres11.png}}
\begin{frame}
\frametitle{}
\vspace{-0.5cm}

\begin{center}
\hspace*{-1cm}\includegraphics[scale=0.25]{imagenes/areaecol1.png}\\

\end{center}

\end{frame}
}
{
\usebackgroundtemplate{\includegraphics[width=\paperwidth,height=\paperheight]{imagenes/dibujpres11.png}}
\begin{frame}
\frametitle{}
\vspace{-0.5cm}

\begin{center}
\hspace*{-1cm}\includegraphics[scale=0.25]{imagenes/areaecol2.png}\\

\end{center}

\end{frame}
}

{
\usebackgroundtemplate{\includegraphics[width=\paperwidth,height=\paperheight]{imagenes/dibujpres11.png}}
\begin{frame}
\frametitle{}
\begin{center}
{\normalsize
\begin{tabular}{ccc} \toprule
Área verde & Puntos \\ \midrule
Parque Laguna de Chapulco & 6 \\
Ciudad universitaria BUAP & 20 \\
Los Fuertes & 20 \\
Panteón Municipal & 8  \\
Parque del Arte & 5 \\
Parque del Atoyac & 12 \\
Parque Ecológico & 14 \\
Parque Juárez & 3 \\
Parque Paseo bravo & 3 \\ \bottonrule
 ×
\end{tabular}
}
\end{center}
\end{frame}
}
{
\usebackgroundtemplate{\includegraphics[width=\paperwidth,height=\paperheight]{imagenes/dibujpres11.png}}
\begin{frame}
\frametitle{}
\vspace{-0.5cm}

\begin{center}
\hspace*{-1cm}\includegraphics[scale=0.25]{imagenes/disenin.png}\\

\end{center}

\end{frame}
}



%%%los libros murieron
%{
%\usebackgroundtemplate{\includegraphics[width=\paperwidth,height=\paperheight]{imagenes/dibujpres11.png}}
%\begin{frame}
%\frametitle{}
%\begin{itemize}
%\small{ \item Temporada migratoria (enero-marzo) y la reproductiva(abril-junio).}
%\small{ \item Cuatro veces (dos veces por temporada).}
%\small{ \item Desde  8:00 am hasta terminar los puntos de conteo.}\\[-8cm]
%\end{itemize}
%\begin{columns}[t]
%\column{.3\textwidth}
%\centering
%\includegraphics[scale=0.45]{imagenes/Internet/van.jpg}\\
%\column{.3\textwidth}
%\centering
%\includegraphics[scale=0.4]{imagenes/Internet/howell.jpg}\\
%\column{.3\textwidth}
%\centering
%\includegraphics[scale=0.25]{imagenes/Internet/colibries.jpg}
%\end{columns}
%\begin{itemize}
%\small{ \item American Ornithologist Union (AOU 2014). }
%\small{ \item International Ornithological Committee (IOC 2014).}
%\end{itemize}
%\end{frame}
%}

%{
%\usebackgroundtemplate{\includegraphics[width=\paperwidth,height=\paperheight]{imagenes/dibujpres11.png}}
%\begin{frame}
%\frametitle{}
%\begin{itemize}
%\small{ \item La estacionalidad y los gremios de las especies las definí con la guía Ehrlich y colaboradores (1988) .}
%\end{itemize}
%\begin{center}
%\includegraphics[scale=0.25]{imagenes/Internet/erlich.jpg}

%\end{center}
%\end{frame}
%}
%%%%%%%%%%%%%los libros murieron
%{
%\usebackgroundtemplate{\includegraphics[width=\paperwidth,height=\paperheight]{imagenes/dibujpres11.png}}
%\begin{frame}
%\frametitle{Análisis de diversidad}
%\vspace{-2cm}
%\hspace{-1cm}
%\begin{center}
%\includegraphics[scale=0.22]{imagenes/diagramadiversi1.png}

%\end{center}
%\end{frame}
%}


{
\usebackgroundtemplate{\includegraphics[width=\paperwidth,height=\paperheight]{imagenes/dibujpres11.png}}
\begin{frame}
\frametitle{Análisis de diversidad y estadísticos }
\framesubtitle{Estimadores y abundancia}
\begin{center}
\includegraphics[scale=0.45]{imagenes/interprue.png}
\end{center}

\end{frame}
}

{
\usebackgroundtemplate{\includegraphics[width=\paperwidth,height=\paperheight]{imagenes/dibujpres11.png}}
\begin{frame}
\frametitle{ }
\begin{center}
\includegraphics[scale=0.45]{imagenes/rankprue.png}
\end{center}

\end{frame}
}
%{
%\usebackgroundtemplate{\includegraphics[width=\paperwidth,height=\paperheight]{imagenes/dibujpres11.png}}
%\begin{frame}
%\frametitle{ }
%\begin{center}
%\includegraphics[scale=0.45]{imagenes/corprue.png}
%\end{center}

%\end{frame}
%}
{
\usebackgroundtemplate{\includegraphics[width=\paperwidth,height=\paperheight]{imagenes/dibujpres11.png}}
\begin{frame}
\frametitle{Diversidad $\alpha$ }
\begin{center}
\includegraphics[scale=0.45]{imagenes/intereje.png}
\end{center}

\end{frame}
}

{
\usebackgroundtemplate{\includegraphics[width=\paperwidth,height=\paperheight]{imagenes/dibujpres11.png}}
\begin{frame}
\frametitle{}
\vspace{-1cm}

\begin{center}
\includegraphics[scale=0.3]{imagenes/ancovilla.png}
\end{center}
\begin{itemize}
\normalsize{\item Diversidad~Tipo de hábitat + Área arbolada}
\normalsize{\item Diversidad~Tipo de hábitat + Distancia rural}
\normalsize{\item Crawley (2012)}

\end{itemize}
\end{frame}
}



%{
%\usebackgroundtemplate{\includegraphics[width=\paperwidth,height=\paperheight]{imagenes/dibujpres11.png}}
%\begin{frame}
%\frametitle{}
%\vspace{-2.5cm}

%\begin{center}
%\includegraphics[height=5.5cm,width=11cm]{imagenes/ancoeje.png}
%\end{center}
%\begin{itemize}
%\normalsize{\item q=1~Tipo de hábitat + Área arbolada}
%\end{itemize}
%\end{frame}
%}

%{
%\usebackgroundtemplate{\includegraphics[width=\paperwidth,height=\paperheight]{imagenes/dibujpres11.png}}
%\begin{frame}
%\frametitle{}
%\vspace{-1.5cm}

%\begin{center}
%\includegraphics[height=5.5cm,width=11cm]{imagenes/ancoeje1.png}
%\end{center}

%\begin{itemize}
%\normalsize{\item q=1~Tipo de hábitat + Distancia rural}
%\normalsize{\item Crawley (2012)}

%\end{itemize}
%\end{frame}
%}

{
\usebackgroundtemplate{\includegraphics[width=\paperwidth,height=\paperheight]{imagenes/dibujpres11.png}}
\begin{frame}
\frametitle{Similitud entre sitios}
%\vspace{-1.5cm}

\begin{center}
\includegraphics[scale=0.5]{imagenes/indiceeje.png}
\end{center}
\vspace{-2cm}

\begin {columns}
\column{0.5\textwidth}
\begin{itemize}
\normalsize{\item <1> $\beta$sim-- Presencia/Ausencia}\\
\normalsize{\item <2> Koleff (2005)}\\
\end{itemize}
\column{0.5\textwidth}
\begin{itemize}

\normalsize{\item  <3> perMANOVA\\
Anderson (2001)}
\normalsize{\item <4> Perfil de diversidad $\beta$\\
Jost (2007)}

\end{itemize}
\end{columns}
\end{frame}
}


{
\usebackgroundtemplate{\includegraphics[width=\paperwidth,height=\paperheight]{imagenes/dibujpres11.png}}
\begin{frame}
\frametitle{Diversidad de gremios}
\vspace{-0.5cm}

\begin{center}
\includegraphics[scale=0.5]{imagenes/gremioeje.png}
\end{center}
\vspace{-1cm}

\begin{itemize}
\pause
\normalsize{\item Tabla de contingencia (Zar 2010)}
\pause
\normalsize{\item Modelos nulos (Morrone 2001)}
\pause
\normalsize{\item 1000 remuestreos}

\end{itemize}
\end{frame}
}

%{
%\usebackgroundtemplate{\includegraphics[width=\paperwidth,height=\paperheight]{imagenes/dibujpres11.png}}
%\begin{frame}
%\frametitle{Análisis de diversidad}
%\tikzstyle{every node}=[draw=black,thick,anchor=west]
%\tikzstyle{selected}=[draw=red,fill=red!30]
%\tikzstyle{optional}=[dashed]
%\onslide<1>{\begin{tikzpicture}[%path={(\tikzparentnode.south) |- (\tikzchildnode.west)}]
  %\onslide<1>{\node(a){Estimadores y abundancia}};
  %\onslide<2>{\node(b) [below=1cm of a]{Intrapolación/extrapolación (Chao 2012)}};
  %\onslide<1>{\node(c) [below=1cm of b]{Estimadores y abundancia}};
 %grow via three points={one child at (0.5,-0.7) and
  %two children at (0.5,-0.7) and (0.5,-1.4)},
  %edge from parent path={(\tikzparentnode.south) |- (\tikzchildnode.west)}]
  %\node {Estimadores y abundancia}
   % child { node {Intrapolación/extrapolación (Chao 2012)}}		
    %child { node {Cobertura de la muestra (Colwell et al 2014)}}
    %child { node {Rango abundancia log 10 (Magurran 2004)	}};

%\end{tikzpicture}}
%\onslide<2>{\begin{tikzpicture}[%
 % {one child at (0.5,-0.7) and
  %two children at (0.5,-0.7) and (0.5,-1.4)},
  %edge from parent path={(\tikzparentnode.south) |- (\tikzchildnode.west)}]
  %\node [optional] {Diversidad $\alpha$ }
   % child { node [optional] {Intervalos de confianza al 84\% \\ (MacGregor-Fors y Payton 2013)}};
%\end{tikzpicture}}
%\end{frame}
%}

%{
%\usebackgroundtemplate{\includegraphics[width=\paperwidth,height=\paperheight]{imagenes/dibujpres11.png}}
%\begin{frame}
%\frametitle{}
%\tikzstyle{every node}=[draw=black,thick,anchor=west]
%\tikzstyle{selected}=[draw=blue]
%\tikzstyle{optional}=[dashed]
%\onslide<1>{\begin{tikzpicture}[
 %grow via three points={one child at (1,-2) and
  %two children at (1,-2) and (1,-3)},
  %edge from parent path={(\tikzparentnode.south) |- (\tikzchildnode.west)}]
  %\node [selected] {Similitud entre comunidades}
   % child { node [selected] {Perfil de diversidad beta (Jost 2007)}}
    %child { node [selected] {Índice $\beta$sim (Koleff 2005)}
%child { node [selected]{Dendrogramas}}
%child { node [selected]{Distancias máximas}}
%};		
%\end{tikzpicture}}

%\end{frame}
%}

%{
%\usebackgroundtemplate{\includegraphics[width=\paperwidth,height=\paperheight]{imagenes/dibujpres11.png}}
%\begin{frame}
%\frametitle{}
%\tikzstyle{every node}=[draw=black,thick,anchor=west]
%\tikzstyle{selected}=[draw=blue]
%\tikzstyle{optional}=[dashed]
%\begin{tikzpicture}[%
 % {one child at (-1,-2) and
 % two children at (3,-6) and (-1,-7)},
  %edge from parent path={(\tikzparentnode.south) |- (\tikzchildnode.west)}]
  %\node [optional] {Diversidad de gremios }
   % child { node [optional] {Dendrogramas distancias euclidianas}}
    %child { node [optional] {Modelo nulo 1000 remuestreos}};
%\end{tikzpicture}
%\end{frame}
%}

%\begin{frame}
 %   \begin{itemize}
  %      \item<1> First
   %     \item<2>Second
    %    \item<3>Third
    %\end{itemize}
  %\end{frame}

%{
%\usebackgroundtemplate{\includegraphics[width=\paperwidth,height=\paperheight]{imagenes/dibujpres11.png}}
%\begin{frame}
%\frametitle{Análisis estadísticos}
%\vspace{-2cm}
%\hspace{-1cm}
%\begin{center}
%\includegraphics[scale=0.22]{imagenes/diagramadiversi2.png}

%\end{center}
%\end{frame}
%}



\section{Resultados y discusión}

{
\usebackgroundtemplate{\includegraphics[width=\paperwidth,height=\paperheight]{imagenes/dibujpres11.png}}
\begin{frame}
\frametitle{Resultados y discusión}
\begin{center}
\includegraphics[scale=0.4]{imagenes/cuantos.png}

\end{center}
\end{frame}
}

{
\usebackgroundtemplate{\includegraphics[width=\paperwidth,height=\paperheight]{imagenes/dibujpres11.png}}
\begin{frame}
\frametitle{Otros estudios...}
\vspace{-0.5cm}

\begin{center}
\hspace*{-1cm}\includegraphics[scale=0.25]{imagenes/mapa2pres.png}\\

\end{center}

\end{frame}
}

%{
%\usebackgroundtemplate{\includegraphics[width=\paperwidth,height=\paperheight]{imagenes/dibujpres11.png}}
%\begin{frame}
%\frametitle{}
%\begin{center}
%\includegraphics[scale=0.45]{imagenes/tipos.png}

%\end{center}
%\end{frame}
%}


%{
%\usebackgroundtemplate{\includegraphics[width=\paperwidth,height=\paperheight]{imagenes/dibujpres11.png}}
%\begin{frame}
%\frametitle{}
%\begin{center}
%\includegraphics[scale=0.45]{imagenes/tipos1.png}

%\end{center}
%\end{frame}
%}

%{
%\usebackgroundtemplate{\includegraphics[width=\paperwidth,height=\paperheight]{imagenes/dibujpres11.png}}
%\begin{frame}
%\frametitle{}
%\vspace{-1cm}
%\begin{center}
%\hspace*{-0.9cm}\includegraphics[scale=0.5]{imagenes/familiaspres.png}
%\end{center}

%\end{frame}
%}
{
\usebackgroundtemplate{\includegraphics[width=\paperwidth,height=\paperheight]{imagenes/dibujpres11.png}}
\begin{frame}
\frametitle{Abundancia}
\begin{columns}[t]
\column{.5\textwidth}
\centering
\includegraphics[scale=0.25]{imagenes/Internet/quiscalus.jpg}\\
\caption{\normalsize{ Zanate (\textit{Quiscalus mexicanus}).}}\\
\column{.5\textwidth}
\centering
\includegraphics[scale=2.3]{imagenes/Internet/passer.jpg}\\
\caption{\normalsize{Gorrión doméstico (\textit{Passer domesticus}).}}
\end{columns}\\[2cm]
\tiny{Fotos tomadas de http://naturalista.conabio.gob.mx }

\end{frame}
}

{
\usebackgroundtemplate{\includegraphics[width=\paperwidth,height=\paperheight]{imagenes/dibujpres11.png}}
\begin{frame}
\frametitle{}
\begin{columns}[t]
\column{.5\textwidth}
\centering
\includegraphics[scale=1.1]{imagenes/Internet/paloma.jpg}\\
\caption{\normalsize{ Paloma común (\textit{Columba livia}).}}\\
\column{.5\textwidth}
\centering
\includegraphics[scale=0.9]{imagenes/Internet/carpodacus.jpg}\\
\caption{\normalsize{ Gorrión mexicano (\textit{Haemorhous mexicanus}).}}
\end{columns}\\[-2cm]
\begin{center}
\centering
\includegraphics[scale=0.3]{imagenes/Internet/coquita2.jpg}\\
\caption{\normalsize{Coquita (\textit{Columbina inca}).}}
\end{center}
\tiny{Fotos tomadas de http://naturalista.conabio.gob.mx }

\end{frame}
}

%{
%\usebackgroundtemplate{\includegraphics[width=\paperwidth,height=\paperheight]{imagenes/dibujpres11.png}}
%\begin{frame}
%\frametitle{}
%\vspace{-1.5cm}
%\begin{center}
%\centering
%\includegraphics[scale=0.5]{imagenes/unidup.png}\\
%\end{center}
%\end{frame}
%}
%{
%\usebackgroundtemplate{\includegraphics[width=\paperwidth,height=\paperheight]{imagenes/dibujpres11.png}}
%\begin{frame}
%\frametitle{}
%\vspace{-1.5cm}
%\begin{columns}[t]
%\column{.5\textwidth}
%\begin{itemize}
%\normalsize{\item Únicas}
%\end{itemize}
%\centering
%\includegraphics[scale=0.2]{imagenes/Internet/cather.jpg}\\
%\caption{\normalsize{ Chivirín barranqueño (\textit{Catherpes mexicanus}).}}\\
%\includegraphics[scale=0.9]{imagenes/Internet/mnioti.jpg}\\
%\caption{\normalsize{ Chipe trepador (\textit{Mniotilta varia}).}}\\
%\column{.5\textwidth}
%\begin{itemize}
%\normalsize{\item Duplicados}
%\end{itemize}
%\centering
%\includegraphics[scale=1.2]{imagenes/Internet/sayo.jpg}\\
%\caption{\normalsize{ Mosquerito negro (\textit{Sayornis nigricans}).}}\\
%\includegraphics[scale=0.2]{imagenes/Internet/geo.jpg}\\
%\caption{\normalsize{ Máscarita común (\textit{Geothlypis trichas}).}}\\

%\end{columns}\\[-2cm]
%\tiny{Fotos tomadas de http://naturalista.conabio.gob.mx }

%\end{frame}
%}
{
\usebackgroundtemplate{\includegraphics[width=\paperwidth,height=\paperheight]{imagenes/dibujpres11.png}}
\begin{frame}
\frametitle{Ecología urbana}
\vspace{-0.5cm}

\begin{center}
\hspace*{-1cm}\includegraphics[scale=0.25]{imagenes/fischer.png}\\

\end{center}

\end{frame}
}

{
\usebackgroundtemplate{\includegraphics[width=\paperwidth,height=\paperheight]{imagenes/dibujpres11.png}}
\begin{frame}
\frametitle{Ecología urbana}
\vspace{-0.5cm}

\begin{center}
\hspace*{-1cm}\includegraphics[scale=0.25]{imagenes/fischer1.png}\\

\end{center}

\end{frame}
}


{
\usebackgroundtemplate{\includegraphics[width=\paperwidth,height=\paperheight]{imagenes/dibujpres11.png}}
\begin{frame}
\frametitle{}
\begin{itemize}
\normalsize{ \item  NOM-059-2010  categoría de \textbf{protección especial}.}
\end{itemize}
\begin{columns}[t]
\column{.5\textwidth}
\centering
\includegraphics[scale=0.5]{imagenes/Internet/xenotriccus1.jpg}\\
\caption{\tiny{ Mosquero del Balsas (\textit{Xenotriccus mexicanus}).}}\\
\column{.5\textwidth}
\centering
\includegraphics[scale=1.1]{imagenes/Internet/Myadestes.jpg}\\
\caption{\tiny{ Clarín jilguero (\textit{Myadestes occidentalis}).}}
\end{columns}\\[2cm]
\tiny{Fotos tomadas de http://naturalista.conabio.gob.mx }

\end{frame}
}

{
\usebackgroundtemplate{\includegraphics[width=\paperwidth,height=\paperheight]{imagenes/dibujpres11.png}}
\begin{frame}
\frametitle{}
\begin{columns}[t]
\column{.5\textwidth}
\centering
\includegraphics[scale=0.4]{imagenes/Internet/cooperii.jpg}\\
\caption{\tiny{ Halcón de Cooper* (\textit{Accipiter cooperii})}}\\
\column{.5\textwidth}
\centering
\includegraphics[scale=0.6]{imagenes/Internet/striatus.jpg}\\
\caption{\tiny{ Gavilan de pecho rufo (\textit{Accipites striatus}).}}
\end{columns}\\[2cm]
\tiny{Fotos tomadas de http://naturalista.conabio.gob.mx y http://avesmx.conabio.gob.mx*}

\end{frame}
}


{
\usebackgroundtemplate{\includegraphics[width=\paperwidth,height=\paperheight]{imagenes/dibujpres11.png}}
\begin{frame}
\frametitle{Áreas verdes y grises}
\vspace{-1cm}
\begin{center}
\hspace*{-0.9cm}\includegraphics[scale=0.55]{imagenes/sitiospres.png}\\
\end{center}
\end{frame}
}


{
\usebackgroundtemplate{\includegraphics[width=\paperwidth,height=\paperheight]{imagenes/dibujpres11.png}}
\begin{frame}
\frametitle{}
\vspace{-1.8cm}
\begin{center}
\hspace*{-1.2cm}\includegraphics[scale=0.43]{imagenes/rangopres2.png}\\
\end{center}
\end{frame}
}

{
\usebackgroundtemplate{\includegraphics[width=\paperwidth,height=\paperheight]{imagenes/dibujpres11.png}}
\begin{frame}
\frametitle{}
\vspace{-1cm}
\begin{center}
\includegraphics[scale=0.43]{imagenes/corpre.png}\\
\end{center}
\end{frame}
}

{
\usebackgroundtemplate{\includegraphics[width=\paperwidth,height=\paperheight]{imagenes/dibujpres11.png}}
\begin{frame}
\frametitle{}
\begin{center}
\includegraphics[scale=0.42]{imagenes/estimapres.png}\\
\end{center}
\vspace{-0.5cm}
{\tiny
\begin{longtable}[c] {lllllr} 
\caption{} \\
Sitio &CM & R obs  & R est & U & D   \\ \midrule
Total  & 99\% & 77 & 98 & 13 & 4 \\
Áreas verdes  & 99\% & 74 & 85.99 & 12 & 6 \\
Áreas grises  & 99\% & 40 & 63.997 & 12 & 3 \\ \bottomrule
\end{longtable}
}\\[5cm]

\end{frame}
}

{
\usebackgroundtemplate{\includegraphics[width=\paperwidth,height=\paperheight]{imagenes/dibujpres11.png}}
\begin{frame}
\frametitle{Diversidad $\alpha$}
\begin{center}
{\tiny
\begin{longtable}[c] {lllllr} 
\caption[]{} \\
 Término & gl & SC & CM & F & \textit{P} \\ \midrule
Área arbolada (A) &1& 179.696 & 179.696 & 38.02 & <<0.0001 \\
Sitios(S) & 1 & 1.001  & 1.001 & 0.211 & 0.652 \\
AxS &1& 0.029 & 0.029 & 0.006 & 0.938 \\ 
Error &14& 66.163 & 4.726  \\
Total &17& 246.86 & 14.523 \\ \bottomrule 
\end{longtable}
}
\end{center}\\
\begin{center}
{\tiny
\begin{longtable}[c] {lllllr} 
\caption[]{} \\
 Término &gl& SC & CM & F & \textit{P} \\ \midrule
Distancia (D) &1& 7.966 & 7.966 & 1.8868 & 0.1912  \\
Sitios(S)  &1& 179.696 & 179.696 &42.5628& <<0.0001 \\
DxS &1& 0.121 & 0.121 & 0.0287 & 0.867  \\ 
Error & 14 & 59.107 & 4.222 \\
Total &17& 246.86 & 14.523  \\ \bottomrule 
\end{longtable}
} 
\end{center}
\end{frame}
}

{
\usebackgroundtemplate{\includegraphics[width=\paperwidth,height=\paperheight]{imagenes/dibujpres11.png}}
\begin{frame}
\frametitle{}
\begin{center}
\includegraphics[scale=0.6]{imagenes/raref84pres.png}\\
\end{center}

\end{frame}
}

{
\usebackgroundtemplate{\includegraphics[width=\paperwidth,height=\paperheight]{imagenes/dibujpres11.png}}
\begin{frame}
\frametitle{Similitud entre comunidades}
\vspace{-0.8cm}
\begin{center}
\includegraphics[scale=0.55]{imagenes/betapsimpres.png}\\
\end{center}

\end{frame}
}

{
\usebackgroundtemplate{\includegraphics[width=\paperwidth,height=\paperheight]{imagenes/dibujpres11.png}}
\begin{frame}
\frametitle{}
\begin{center}
\includegraphics[scale=0.5]{imagenes/betacsimpres.png}\\
\end{center}

\end{frame}
}

{
\usebackgroundtemplate{\includegraphics[width=\paperwidth,height=\paperheight]{imagenes/dibujpres11.png}}
\begin{frame}
\frametitle{}
\begin{itemize}
\normalsize{\item perMANOVA  (\textit{P}=0.01, R=0.226).}
\end{itemize}
\begin{center}
{\small
\begin{longtable}[c] {ccccccc} 
\caption[]{} \\
Término & gl & SC & CM & F & R & \textit{P}   \\ \midrule
Habitat & 17 & 34.248 & 2.014 & 13.428 & {\color{red} 22.6\%} & 0.01 \\
Visitas & 3 & 9.790 & 3.263 & 21.751 & {\color{red} 6.4\%} & 0.01 \\
Residuales & 717 & 107.268 & 0.15 &  &{\color{red} 70.8\%} &  \\ 
Total & 735 & 151.305 & & & 1 & \\ \bottomrule

\end{longtable}
}
\end{center}

\end{frame}
}

{
\usebackgroundtemplate{\includegraphics[width=\paperwidth,height=\paperheight]{imagenes/dibujpres11.png}}
\begin{frame}
\frametitle{}
\begin{center}
\includegraphics[scale=0.7]{imagenes/perfilib1.png}\\
\end{center}
\end{frame}
}
{
\usebackgroundtemplate{\includegraphics[width=\paperwidth,height=\paperheight]{imagenes/dibujpres11.png}}
\begin{frame}
\frametitle{Diversidad de gremios}

\begin{center}
\hspace*{-0.5cm}\includegraphics[scale=0.5]{imagenes/grems.png}\\
\end{center}

\begin{center}
\begin{itemize}
\centering
\normalsize{\item $\chi^{2}$=2.697,gl=4,\textit{P}=0.61}
\end{itemize}
\end{center}
\end{frame}
}

{
\usebackgroundtemplate{\includegraphics[width=\paperwidth,height=\paperheight]{imagenes/dibujpres11.png}}
\begin{frame}
\frametitle{}
\vspace{-1.5cm}
\begin{center}
\hspace*{-2cm}\includegraphics[scale=0.35]{imagenes/gremiosprese.png}\\
\end{center}
\end{frame}
}

%{
%\usebackgroundtemplate{\includegraphics[width=\paperwidth,height=\paperheight]{imagenes/dibujpres11.png}}
%\begin{frame}
%\frametitle{}
%\begin{itemize}
%\tiny{\item Sin embargo el número de especies insectívoras disminuye en las áreas grises.}
%\end{itemize}
%\vspace{-0cm}
%\begin{center}
%\includegraphics[scale=0.45]{imagenes/Figura14.png}\\
%\end{center}
%\end{frame}
%}

{
\usebackgroundtemplate{\includegraphics[width=\paperwidth,height=\paperheight]{imagenes/dibujpres11.png}}
\begin{frame}
\frametitle{}
\vspace{-0cm}
\begin{center}
\includegraphics[scale=0.3]{imagenes/nulpar.png}\\
\end{center}
\end{frame}
}
{
\usebackgroundtemplate{\includegraphics[width=\paperwidth,height=\paperheight]{imagenes/dibujpres11.png}}
\begin{frame}
\frametitle{}
\begin{center}
\includegraphics[scale=0.3]{imagenes/nulcal.png}\\
\end{center}
\end{frame}
}



\section{Conclusiones}
{
\usebackgroundtemplate{\includegraphics[width=\paperwidth,height=\paperheight]{imagenes/dibujpres11.png}}
\begin{frame}
\frametitle{Conclusiones}
\vspace{-1cm}
\begin{enumerate}
\normalsize{\item  Las áreas verdes son más diversas que las áreas grises en la ciudad de Puebla.
Sin embargo, las áreas grises de la ciudad de Puebla, presentan un número importante de especies de aves que resulta mayor que otras regiones del país.}\\[1cm]
\pause
\normalsize{\item El análisis de gremios muestra estructuras diferentes para las áreas verdes y las áreas grises, donde el número de especies de aves insectívoras decrece en las áreas grises a comparación de las áreas verdes.}\\[1cm]
\pause
\normalsize{\item El análisis de  diversidad beta para áreas verdes y grises indica  que existe una sola comunidad de aves.}
\end{enumerate}

\end{frame}
}
{
\usebackgroundtemplate{\includegraphics[width=\paperwidth,height=\paperheight]{imagenes/dibujpres10.png}}

\begin{frame}
\frametitle{}
\Huge{ Quisiera agradecer a......}
\end{frame}
}

{
\usebackgroundtemplate{\includegraphics[width=\paperwidth,height=\paperheight]{imagenes/dibujpres10.png}}

\begin{frame}
\frametitle{}
\begin{itemize}
\small{\item \textit{Mi familia}}
\pause
\small{\item \textit{Dr Juan Héctor García Chávez}}\\
\pause
\small{\item \textit{M en C Ana Lucia Castillo Meza}}\\
\pause
\small{\item \textit{A mis sinodales Dr Antonio Fernández Crispín\\
Dr Federico Escobar Sarria\\
Dr Ian MacGregor-Fors}}
\pause
\small{\item \textit{Gabriela Angulo Ordoñes}}\\
\pause
\small{\item \textit{A mis amigos}}\\


\end{itemize}

\end{frame}
}
{
\setbeamertemplate{headline}{}
{
\usebackgroundtemplate{\includegraphics[width=\paperwidth,height=\paperheight]{imagenes/disenin2.png}}

\begin{frame}
\frametitle{}
%\begin{figure}[ht]
 %  \movie[width=9.1cm,height=6.5cm,showcontrols=true,loop,poster,text={\small(Loading Video...)}]{}{imagenes/Internet/COol gif.avi}
%\end{figure}
%\includemovie[poster,autoplay,externalviewer, text={\small(Loading bloch.mp4)}]{6cm}{6cm}{imagenes/Internet/COol gif.mp4}

%\movie[width=3cm,height=2cm,poster]{}{imagenes/Internet/COol gif.gif}%como lo sugiere la guía
%\includemovie{3cm}{2cm}{imagenes/Internet/COol gif.gif}
%\movie{\pgfuseimage{myposterimage}}{imagenes/Internet/COol gif.avi}
%\movie[autostart]{}{imagenes/Internet/COol gif.mp4}
\end{frame}
}
}

\end{document}

